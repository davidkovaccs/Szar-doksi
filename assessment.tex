%----------------------------------------------------------------------------
\chapter*{Értékelés}\label{sect:ertekeles}
%----------------------------------------------------------------------------

A házi feladat elkészítése során végigvittük egy alkalmazás megvalósítását az abasztrakt leírástól a működő alkalmazásig és az őt leíró dokumentáció elkészítéséig. A kezdeti személyes kontaktus és az előzetes megbeszélések után kihasználtuk a GIT verziókezelő és hálózati kommunikációs technológiák előnyeit, így távolról tudtunk dolgozni a közös projekten, miközben folyamatos virtuális kapcsolatban voltunk egymással. A munka során a követelményspecifikációban lefektetett terveket és elképzeléseket sikerült megvalósítani, megismerkedtünk néhány új fogalommal, az eddigi tudásunkat pedig tovább mélyítettük a már ismert technológiákkal kapcsolatban is. Remélhetőleg az itt szerzett tapasztalatokat későbbi munkáink során hatékonyan kamatoztathatjuk, amennyiben a mostanihoz hasonló komplex tervezést, együttműködést és kiviteleyést igénylő feladatokkal találjuk magunkat szembe.

%,,,,,,,,,,,,,,,,,,,,,,,,,,,,,,,,,,,,,,,,,,,,,,,,,,,,,,,,,,,,,,,,,,,,,,,,,,,,
\subsubsection{Továbbfejlesztési lehetőségek}\label{sect:tovabbfejlesztes}
%,,,,,,,,,,,,,,,,,,,,,,,,,,,,,,,,,,,,,,,,,,,,,,,,,,,,,,,,,,,,,,,,,,,,,,,,,,,,

A szoftver jelenleg az aktuális árfolyamok számítására egy egyszerű modellt használ, mely a valóságban jóval bonyolultabban működik. A valós adatok részletes tanulmányozásával valószínüleg jóval komplexebb és reálisabb leíró modellt tudnánk alkotni, amely meghatározza a tőzsdei árfolyamok változásának mechanizmusait. 
Jeleneg egy egyszerű, de átlátható kezelőfelülettel rendelkezik az alkalmazás. Egy hivatalos designer egyedi és divatos megjelenést szabhatna az oldalnak, mellyel fokozhatná mind a használhatóságot, mind a kényelmet, mind az esztétikai élvezetet az oldal használata közben.
A megoldás nincsen felkészülve nagy tömegek kiszolgálására, nincs optimalizálva a válaszidők minimalizálására és kiélezve a hatékonyságra. Minden olyan egyszerűen és egyértelműen van megoldva, hogy átlátható és redundanciától lehetőleg mentes legyen, de egy teljesítmánykritikus rendszer esetében sok optimalizációs lehetőség rendelkezésünkre áll. 