%----------------------------------------------------------------------------
\chapter{Rendszerterv}\label{sect:rszterv}
%----------------------------------------------------------------------------

Jelen fejezetben szeretnénk bemutatni az elkészítendő rendszer tervét. A következő részek kitérnek az architektúra kiválasztására és megvalósítására (\sectref{architektura} szakasz), az itt szereplő rétegek felépítésére és funkciójára (\sectref{retegek} szakasz), valamint végül a program funkcióinak ismertetésére (\sectref{rsz_funkciok} szakasz).

%,,,,,,,,,,,,,,,,,,,,,,,,,,,,,,,,,,,,,,,,,,,,,,,,,,,,,,,,,,,,,,,,,,,,,,,,,,,,
\section{Architektúra}\label{sect:architektura}
%,,,,,,,,,,,,,,,,,,,,,,,,,,,,,,,,,,,,,,,,,,,,,,,,,,,,,,,,,,,,,,,,,,,,,,,,,,,,

Az alkalmazás három rétegből épül fel a klasszikus MVC architektúra alapján. A kliens oldalon a felhasználó webböngészőben futtatja az alkalmazást. A szerver oldali kód egy Apache Web Serveren fog futni, mely többszálú kliens kiszolgálást tesz lehetővé és képes titkosított csatornán végezni a kommunikációt a köztes rétegben, nagyban növelve a privát adatok védelmét. Az modell réteg a PostgreSQL adatbázis kezelő rendszeren alapul, mely egy széles körben támogatott open-source megoldás, mely megfelelő hatékonységgal képes szolgálni a rendszert. 

%. . . . . . . . . . . . . . . . . . . . . . . . . . . . . . . . . . . . . .
%\subsubsection{Egy subsubsection}\label{sect:egysubsubsection}
%. . . . . . . . . . . . . . . . . . . . . . . . . . . . . . . . . . . . . .

%,,,,,,,,,,,,,,,,,,,,,,,,,,,,,,,,,,,,,,,,,,,,,,,,,,,,,,,,,,,,,,,,,,,,,,,,,,,,
\section{Rétegek}\label{sect:retegek}
%,,,,,,,,,,,,,,,,,,,,,,,,,,,,,,,,,,,,,,,,,,,,,,,,,,,,,,,,,,,,,,,,,,,,,,,,,,,,

 A megjelenítési réteg a kliens oldalon játszik fő szerepet, ahol a rendszer felhasználója egy tetszőleges böngészőben futtathatja a webalkalmazást. Az üzleti logikai funckiók a szerver oldalon futnak, a webalkalmazáshoz egy dedikált szerver tartozik, mely képes több kliens egyidejű kiszolgálására. Ide érkeznek be kliens oldalról a kérések, az üzleti logikai rétegen keresztül tudunk szolgáltatásokat nyújtani a kliens felé. A felhasználók perzisztens adataira, a különböző megbízások és ügyletek adatainak tárolására egy adatbázis és a programozható kezelőfelülete áll rendelekezésünkre.

%............................................................................
\subsection{Megjelenítési réteg}\label{sect:kliens_reteg}
%............................................................................

Az felhasználó számítógépén, egy tetszőleges böngészőben fut az alkalmazás kliens rétege, ami a vizuális megjelenítésért felelős és kapcsolódásokat definiál az üzleti logika réteg felé.

Az alkalmazás a legmodernebb webes trendeket követi, szem előtt tartva a minél nagyobb körű kompatibiltást különböző böngészőkkel. A felhasználóhoz érkező végső termék egy HTML lap CSS stíluslapokkal és Javascript kódokkal megtámogatva. 
A CSS fájlok felelnek a felhaszálói felületen elhelyezett elemek beállításáért, segítséget nyújtanak egy koherens design megalkotásában, ahol az elemek vizuális tulajdonságai teljes mértékben lecsatolódnak az elemek definiálásától.
A HTML kódban hozunk létre elemeket a kliens alkalmazás felhasználói felületén. Az elemek lehetnek statikus vagy interaktív elemek, utóbbiakkat használva a felhasználó az üzleti logika felé kérésekkel fordulhat, melyekre a szerver válaszol egy újabb HTML fájllal. A felhasználó nem csak újabb oldalakra léphet tovább, hanem adatokat is küldhet, az üzleti logika által nyújtott felületen keresztül manipulálhatja a modell rétegben található adatokat, ha rendelkezik az ehhez megfelelő jogosultságokkal.

\bigskip

A webalkalmazás a kliens-szerver alapokon nyugvó REST architektúrán alapul, amelynek legfontosabb tulajdonsága, hogy minden fogalmát és objektumát a rendszernek erőforrással reprezentáljuk, melyeken különböző műveleteket végezhetünk, ezzel manipulálva a reprezentációt. A kliens oldalon a webböngésző különböző HTTP metódusokat (GET, POST, PUT, DELETE) küldve végezheti ezeket a műveleteket, melyeket az üzleti logika rész megfelelő funkciókhoz irányít.

%............................................................................
\subsection{Webszerver réteg}\label{sect:szerver_reteg}
%............................................................................

A kontroller réteget egy webszerver valósítja meg, melynek feladata a klienstől beérkező kérések és adatok feldolgozása, adatbázis manipulációk kezdeményezése és a válaszok visszaküldése a kliens réteg felé.

A webszerver legkülső felülete egy routing tábla, mely a klienstől érkező, egyes erőforrásokon végzett kéréseket a megfelelő funkcióhoz rendeli. Ezek a funkciók szintén az üzleti logika rétegben vannak definiálva, a különböző erőforrás reprezentációkhoz megfelelő modellek vannak definiálva. A modelleken az objektum-orientált szemléletet követve attribútumok és funckiók vannak definálva. Az osztályok és attribútumok közvetlen kapcsolatban állnak az adatbázis egyes tábláival, definiálhatunk ezen túl egy-egy, egy-több és több-több kapcsolatokat, melyek az adatbázis rétegben kapcsolótáblák segítségével jelennek meg, de az üzleti logika réteg számára ez a működés transzparens.

%............................................................................
\subsection{Adatbázis réteg}\label{sect:adatbazis_reteg}
%............................................................................

Az adatbázisban az üzleti logika részben definiált osztályok tábla reprezentációi szerepelnek, a köztük levő kapcsolatokat megvalósító kapcsolótáblák és különböző segédtáblák, melyek csupán a működést támogatják. Az adatbázisban egy tábla egy erőforrás osztálynak felel meg, ezen osztályok példányai a megfelelő tábla egy sorában vannak tárolva, az osztályok attribútumainak a tábla oszlopai felelnek meg.

A USER tábla tartalmazza a regisztrált felhasználókat, a jelszavakat lekódolva tároljuk az esetleges sikeres támadás hatásainak tompítása érdekében. A bejelentkezéshez kapcsolatos adatok és a felhasználó egyéb adatai tárolódnak ebben a táblában, továbbá egy flag, mely jelzi, hogy a felhasználó regisztrációját aktiválva van-e.

A ROLE tábla tartalmazza a rendszerben megtalálható szerepeket, az adminisztátor, bróker és felhasználó szerep elérhető jelenleg a rendszerben. A felhasználók és szerepek között több-több kapcsolat van, ezt a USER_ROLE kapcsolótábla valósítja meg. 

Minden felhasználóhoz tartozhat több számla, így az ACCOUNT tábla a számla adatai mellett egy referenciát is tartalmaz a megfelelő felhasználóra.

A STOCK tábla különböző részvényeket reprezentál, mely alapvető adatokat tartalmaz a részvényről (név, aktuális árfolyam, különböző időbélyegek).

A TRANSACTIONS tábla tartalmazza a már megkötött ügyleteket egy referenciával a megfelelő részvény típusra. Ezen kívül különböző időbélyegeket és az ügylet árát tartalmazza.

Az ORDERS tábla a megrendeléseket tartalmazza, egy megrendelés egy számlához tartozik, így a megfelelő oszlopban hivatkozik egy számlára. Ezen kívül egy ügylet referenciát is tartalmaz, mely NULL értéket tartalmaz, ha az ügylet még nem jött létre. A harmadik referencia a megrendelt részvényre mutat, ezentúl tartalmazza a megrendeléshez tartozó egyéb információkat (ár, vétel-eladás flag, időbélyegek).

%,,,,,,,,,,,,,,,,,,,,,,,,,,,,,,,,,,,,,,,,,,,,,,,,,,,,,,,,,,,,,,,,,,,,,,,,,,,,
\section{Funkciók}\label{sect:rsz_funkciok}
%,,,,,,,,,,,,,,,,,,,,,,,,,,,,,,,,,,,,,,,,,,,,,,,,,,,,,,,,,,,,,,,,,,,,,,,,,,,,
