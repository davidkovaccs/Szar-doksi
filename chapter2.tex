%----------------------------------------------------------------------------
\chapter{Rendszerterv}\label{sect:rszterv}
%----------------------------------------------------------------------------

Jelen fejezetben szeretnénk bemutatni az elkészítendő rendszer tervét. A következő részek kitérnek az architektúra kiválasztására és megvalósítására (\sectref{architektura} szakasz), az itt szereplő rétegek felépítésére és funkciójára (\sectref{retegek} szakasz), valamint végül a program funkcióinak ismertetésére (\sectref{rsz_funkciok} szakasz).

%,,,,,,,,,,,,,,,,,,,,,,,,,,,,,,,,,,,,,,,,,,,,,,,,,,,,,,,,,,,,,,,,,,,,,,,,,,,,
\section{Architektúra}\label{sect:architektura}
%,,,,,,,,,,,,,,,,,,,,,,,,,,,,,,,,,,,,,,,,,,,,,,,,,,,,,,,,,,,,,,,,,,,,,,,,,,,,


Az alkalmazás három rétegből épül fel a klasszikus MVC architektúra alapján. A kliens oldalon a felhasználó webböngészőben futtatja az alkalmazást. A szerver oldali kód a Heroku felhőben fog futni, mely többszálú kliens kiszolgálást tesz lehetővé és képes titkosított csatornán végezni a kommunikációt a köztes rétegben, nagyban növelve a privát adatok védelmét. Az modell réteg a PostgreSQL adatbázis kezelő rendszeren alapul, mely egy széles körben támogatott open-source megoldás, mely megfelelő hatékonységgal képes szolgálni a rendszert. 

%. . . . . . . . . . . . . . . . . . . . . . . . . . . . . . . . . . . . . .
%\subsubsection{Egy subsubsection}\label{sect:egysubsubsection}
%. . . . . . . . . . . . . . . . . . . . . . . . . . . . . . . . . . . . . .

%,,,,,,,,,,,,,,,,,,,,,,,,,,,,,,,,,,,,,,,,,,,,,,,,,,,,,,,,,,,,,,,,,,,,,,,,,,,,
\section{Rétegek}\label{sect:retegek}
%,,,,,,,,,,,,,,,,,,,,,,,,,,,,,,,,,,,,,,,,,,,,,,,,,,,,,,,,,,,,,,,,,,,,,,,,,,,,

 A megjelenítési réteg a kliens oldalon játszik fő szerepet, ahol a rendszer felhasználója egy tetszőleges böngészőben futtathatja a webalkalmazást. Az üzleti logikai funckiók a szerver oldalon futnak, a webalkalmazáshoz egy dedikált szerver tartozik, mely képes több kliens egyidejű kiszolgálására. Ide érkeznek be kliens oldalról a kérések, az üzleti logikai rétegen keresztül tudunk szolgáltatásokat nyújtani a kliens felé. A felhasználók perzisztens adataira, a különböző megbízások és ügyletek adatainak tárolására egy adatbázis és a programozható kezelőfelülete áll rendelekezésünkre.

%............................................................................
\subsection{Megjelenítési réteg}\label{sect:kliens_reteg}
%............................................................................

Az felhasználó számítógépén, egy tetszőleges böngészőben fut az alkalmazás kliens rétege, ami a vizuális megjelenítésért felelős és kapcsolódásokat definiál az üzleti logika réteg felé.

\bigskip

Az alkalmazás a legmodernebb webes trendeket követi, szem előtt tartva a minél nagyobb körű kompatibiltást különböző böngészőkkel. A felhasználóhoz érkező végső termék egy HTML lap CSS stíluslapokkal és Javascript kódokkal megtámogatva.

\bigskip 

A CSS fájlok felelnek a felhaszálói felületen elhelyezett elemek beállításáért, segítséget nyújtanak egy koherens design megalkotásában, ahol az elemek vizuális tulajdonságai teljes mértékben lecsatolódnak az elemek definiálásától.

\bigskip

A HTML kódban hozunk létre elemeket a kliens alkalmazás felhasználói felületén. Az elemek lehetnek statikus vagy interaktív elemek, utóbbiakkat használva a felhasználó az üzleti logika felé kérésekkel fordulhat, melyekre a szerver válaszol egy újabb HTML fájllal. A felhasználó nem csak újabb oldalakra léphet tovább, hanem adatokat is küldhet, az üzleti logika által nyújtott felületen keresztül manipulálhatja a modell rétegben található adatokat, ha rendelkezik az ehhez megfelelő jogosultságokkal.

\bigskip

A webalkalmazás a kliens-szerver alapokon nyugvó REST architektúrán alapul, amelynek legfontosabb tulajdonsága, hogy minden fogalmát és objektumát a rendszernek erőforrással reprezentáljuk, melyeken különböző műveleteket végezhetünk, ezzel manipulálva a reprezentációt. A kliens oldalon a webböngésző különböző HTTP metódusokat (GET, POST, PUT, DELETE) küldve végezheti ezeket a műveleteket, melyeket az üzleti logika rész megfelelő funkciókhoz irányít.

%............................................................................
\subsection{Webszerver réteg}\label{sect:szerver_reteg}
%............................................................................

A kontroller réteget egy webszerver valósítja meg, melynek feladata a klienstől beérkező kérések és adatok feldolgozása, adatbázis manipulációk kezdeményezése és a válaszok visszaküldése a kliens réteg felé.

\bigskip

A webszerver legkülső felülete egy routing tábla, mely a klienstől érkező, egyes erőforrásokon végzett kéréseket a megfelelő funkcióhoz rendeli. Ezek a funkciók szintén az üzleti logika rétegben vannak definiálva, a különböző erőforrás reprezentációkhoz megfelelő modellek vannak definiálva. A modelleken az objektum-orientált szemléletet követve attribútumok és funckiók vannak definálva. Az osztályok és attribútumok közvetlen kapcsolatban állnak az adatbázis egyes tábláival, definiálhatunk ezen túl egy-egy, egy-több és több-több kapcsolatokat, melyek az adatbázis rétegben kapcsolótáblák segítségével jelennek meg, de az üzleti logika réteg számára ez a működés transzparens.

%............................................................................
\subsection{Adatbázis réteg}\label{sect:adatbazis_reteg}
%............................................................................

Az adatbázisban az üzleti logika részben definiált osztályok tábla reprezentációi szerepelnek, a köztük levő kapcsolatokat megvalósító kapcsolótáblák és különböző segédtáblák, melyek csupán a működést támogatják. Az adatbázisban egy tábla egy erőforrás osztálynak felel meg, ezen osztályok példányai a megfelelő tábla egy sorában vannak tárolva, az osztályok attribútumainak a tábla oszlopai felelnek meg.

\bigskip

A USER tábla tartalmazza a regisztrált felhasználókat, a jelszavakat lekódolva tároljuk az esetleges sikeres támadás hatásainak tompítása érdekében. A bejelentkezéshez kapcsolatos adatok és a felhasználó egyéb adatai tárolódnak ebben a táblában, továbbá egy flag, mely jelzi, hogy a felhasználó regisztrációját aktiválva van-e.

\bigskip

A ROLE tábla tartalmazza a rendszerben megtalálható szerepeket, az adminisztátor, bróker és felhasználó szerep elérhető jelenleg a rendszerben. A felhasználók és szerepek között több-több kapcsolat van, ezt a USER\_ROLE kapcsolótábla valósítja meg. 

\bigskip

Minden felhasználóhoz tartozhat több számla, így az ACCOUNT tábla a számla adatai mellett egy referenciát is tartalmaz a megfelelő felhasználóra.

\bigskip

A STOCK tábla különböző részvényeket reprezentál, mely alapvető adatokat tartalmaz a részvényről (név, aktuális árfolyam, különböző időbélyegek).

\bigskip

A STOCK\_VOL tábla az egyes számlákhoz tartozó részvények számát tárolja, így tárolja a számla és a részvény azonosítóját, illetve egy mennyiséget, ami a részvények számát jelzi.

\bigskip

A TRANSACTIONS tábla tartalmazza a már megkötött ügyleteket egy referenciával a megfelelő részvény típusra. Ezen kívül különböző időbélyegeket és az ügylet árát tartalmazza.

\bigskip

Az ORDERS tábla a megrendeléseket tartalmazza, egy megrendelés egy számlához tartozik, így a megfelelő oszlopban hivatkozik egy számlára. Ezen kívül egy ügylet referenciát is tartalmaz, mely NULL értéket tartalmaz, ha az ügylet még nem jött létre. A harmadik referencia a megrendelt részvényre mutat, ezentúl tartalmazza a megrendeléshez tartozó egyéb információkat (ár, vétel-eladás flag, időbélyegek).

%,,,,,,,,,,,,,,,,,,,,,,,,,,,,,,,,,,,,,,,,,,,,,,,,,,,,,,,,,,,,,,,,,,,,,,,,,,,,
\section{Biztonság és skálázhatóság}\label{sect:rsz_funkciok}
%,,,,,,,,,,,,,,,,,,,,,,,,,,,,,,,,,,,,,,,,,,,,,,,,,,,,,,,,,,,,,,,,,,,,,,,,,,,,
A rendszert a Heroku web kiszolgáló felhőjébe telepítettük, így egyszerre kaptunk nagy skálázhatóságot (tetszőleges méretű adatbázis, tetszőleges mennyisgű adatbázis szál, illetve HTTP kérés kiszolgáló), illetve biztonságot (HTTPS-en keresztüli elérés).

\bigskip

A rendszerben lévő modellekhez kizárólag authentikáció után lehet hozzáférni, illetve deklaratívan le van fektetve melyik felhasználói szerepkör melyik modellhez, milyen formán (létrehozás, módosítás, listázás, törlés) férhet hozzá. Az SQL injection támadások ellen a Rails beviteli mezői automatikusan védenek.

%,,,,,,,,,,,,,,,,,,,,,,,,,,,,,,,,,,,,,,,,,,,,,,,,,,,,,,,,,,,,,,,,,,,,,,,,,,,,
\section{Funkciók}\label{sect:rsz_funkciok}
%,,,,,,,,,,,,,,,,,,,,,,,,,,,,,,,,,,,,,,,,,,,,,,,,,,,,,,,,,,,,,,,,,,,,,,,,,,,,

\subsection{Regisztráció kérelme}
A felhasználó egy felületen a szükséges adatok bevitelével regisztrációt kérvényez (név, email cím, jelszó). A felhasználó ekkor még inaktív állapotban lesz, az adminisztrátornak kell aktiválnia, illetve a megfelelő szerepkört hozzárendelnie.
\newline \underline{Szerepkör}: Ügyfél, Adminisztrátor, Bróker

\subsection{Felhasználó regisztrációja}
A felhasználó a beérkezett regisztrációs kérelmet feldolgozza és létrehoz egy új felhasználót, a role alapvetően ügyfél típusú lesz, de az adminisztrátor átállíthatja. Regisztráció után a felhasználó inaktív állapotban lesz, az adminisztrátornak aktiválnia kell belépéshez.
\newline \underline{Szerepkör}: Adminisztrátor

\subsection{Ügyfél számla kezelése}
Az ügyfél létrehozhat magának tetszőleges mennyiségű számlát amire pénzt tölthet, majd ezt szerkesztheti (pénzt tölt fel, vesz le), törölheti, listázhatja. A számlát aktiválhatja ekkor az lesz az alapértelmezett számla minden művelethez.
\newline \underline{Szerepkör}: Ügyfél

\subsection{Megbízás kezelése}
 A felhasználó kitölt egy megbízási formulát, melyhez megadja a szükséges adatokat(számla, részvény, ár, megbízás tipusa: vétel/eladás), majd ezt beregisztrálja a rendszerbe. A megbízást addig módosíthatja a felhasználó amíg nem lesz belőle ügylet, addig akár törölheti is. Vételi megbízáskor a szükséges mennyiségű pénz azonnal eltávolítódik a számlárol, csak akkor kerül vissza, ha a megbízás törlődik. Eladási megbízásnál az ügylet létrejöttekor kerül a pénz a számlára (ahonnan a részvény véve lett). Megbízást indíthatunk a főoldalról is az 'Eladás' vagy 'Vétel' gomb megnyomásával, ekkor az aktuális árfolyamon az aktivált számláról indul a megbízás.
\newline \underline{Szerepkör}: Ügyfél

\subsection{Részvények árfolyamának megtekintése}
A fő oldalon a felhasználók láthatják a tőzsdén lévő összes részvény múltbéli árát egy grafikon segítségével. A bal oldali listából választhatják ki, hogy éppen melyik részvényt mutassa a grafikon.
\newline \underline{Szerepkör}: Ügyfél

\subsection{Ügyfelek kezelése}
A brókerek látják az összes felhasználót, listázhatják számláikat, megbízásaikat, ügyleteiket. Megbízásaikat akár módosíthatják is.
\newline \underline{Szerepkör}: Bróker
